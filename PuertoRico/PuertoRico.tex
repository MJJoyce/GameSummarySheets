\documentclass[12pt]{article}

% Set landscape mode and custom margins for page while including space for the footer
\usepackage[includefoot, margin={0.5cm, 0.5cm}, landscape]{geometry}

% Used to reduce the spacing around section headers and titles.
% Format title headers by adjusting font size
% Set the spacing on all sides of the title to 0
\usepackage[compact]{titlesec}
\titleformat{\section}{\normalfont\bfseries}{\thesection}{1em}{}
\titlespacing*{\section}{0pt}{*0}{0pt}

\titleformat{\subsection}{\normalfont\bfseries}{\thesection}{0.3em}{}
\titlespacing*{\subsection}{0pt}{*0}{0pt}

% Reduce spacing around list items
\usepackage{enumitem}
\setlist{nolistsep}

% Set line spacing
\usepackage{setspace}
\singlespacing

% Create a multicolumn layout
% Set the amount of separation between columns
% Draw a vertical rule between the columns
\usepackage{multicol}
\setlength{\columnsep}{1cm}
\setlength{\columnseprule}{0.1pt}

% Add a frame around the content
\usepackage{mdframed}

% Custom footer (and header if I wanted)
\usepackage{fancyhdr}
\pagestyle{fancy}

% Set custom headers and footers for fancyhdr
\fancyhead{}
\fancyfoot[C]{Puerto Rico Summary Sheet v1.0}
\renewcommand{\headrulewidth}{0pt} 	% Remove horizontal rule from header
\renewcommand{\footrulewidth}{0pt}  % Remove horizontal rule from footer

% Custom frame style for mdframed
% The negative margin is needed to fix a weird spacing that I couldn't figure out
\mdfdefinestyle{customFrame}{%
    outerlinewidth = 0.4pt,
    innertopmargin = -0.3cm}

% Reduce whitespace in the enumerate list environment
\newenvironment{enumerateCustom}
{\begin{enumerate}
  \setlength{\itemsep}{1pt}
  \setlength{\parskip}{0pt}
  \setlength{\parsep}{0pt}}
{\end{enumerate}}

% Reduce whitespace in the itemize list environment
\newenvironment{itemizeCustom}
{\begin{itemize}
  \setlength{\itemsep}{1pt}
  \setlength{\parskip}{0pt}
  \setlength{\parsep}{0pt}}
{\end{itemize}}

\begin{document}
\begin{multicols*}{2}

\section*{Setup}
\begin{enumerateCustom}
	\item Place out the game board and place all the buildings on their assigned spaces.
	\item Place doubloons on the game board bank.
	\item Give each player 1 player board and 1 less doubloon than there are players.
	\item Pick a starting player and give them the \textbf{Governor card} and a blue \emph{indigo} tile. Give the other players:
		\begin{itemizeCustom}
			\item \textbf{3 Players:} 2nd: indigo / 3rd: corn
			\item \textbf{4 Players:} 2nd: indigo / 3rd and 4th: corn
			\item \textbf{5 Players:} 2nd and 3rd: indigo / 4th and 5th: corn
		\end{itemizeCustom}
	\item Set out victory point chips. 3/4/5 players : 75/100/All points
	\item Set out all 8 query tiles face-up
	\item Shuffle all remaining plantation tiles and place face-down in stacks
	\item Place 1 more plantation tile than the number of players face-up
	\item Place role cards based on number of players. 3 players: All cards but both prospectors / 4 players: All cards but one prospector / 5 players: All cards
	\item Place cargo ships based on number of players. 3/4/5 players : 4-6/5-7/6-8 cargo spaces
	\item Place out goods pieces and the trading house tile
	\item Place colonist ship with <number of player> colonists on it
	\item Place colonists out based on number of players. 3/4/5 players : 55/75/95 colonists
\end{enumerateCustom}

\section*{Goal of the Game}
Players build plantations and buildings, produce goods, and then sell or ship them. Be the player with the most points at the end of the game.

\section*{Game Flow}
Each round, the \textbf{governor} begins by taking an available role card. All players take the chosen action in clockwise order. The next player chooses a role and all players take that action as before. The person who chose the role gets the \emph{privilege} listed on the card. After all players have picked a role, place 1 doubloon on the remaining unused role cards. Return all picked role cards back to the table and pass the \textbf{governor} card to the next player in clockwise order.

\section*{Roles}
\subsection*{The Settler}
Each player takes and places a plantation tile. \textbf{Privilege:} The settler may take and place a quarry \emph{instead}. At the end of the phase, remove untaken plantation tiles and draw replacements.

\subsection*{The Mayor}
Each player takes colonists in player order from the colonist ship and then places \emph{all their colonists} on any empty spaces on their player board. Any colonist(s) that can't be placed may be stored on San Juan for a future Mayor phase. \textbf{Privilege:} The mayor may take an additional colonist from the \emph{supply}. After the phase, the Mayor places 1 colonist on the ship for each empty space on \emph{buildings} of \emph{all} players.

\subsection*{The Builder}
Each player may build a building by paying its cost (left number in the circle). \textbf{Privilege:} The builder pays one less doubloon. Each occupied quarry that a player owns may reduce the cost of building a building by 1 doubloon. Each column of buildings has a maximum number of quarry cost reduction(s) that can be applied. These are rock icons at the top of each column on the game board. The builder's privilege is in addition to the quarry reduction.

\subsection*{The Craftsman}
Each player takes goods from the supply according to his production ability in player order. \textbf{Privilege:} After all players have taken goods, take an additional good (of those you can produce) from the supply.

\subsection*{The Trader}
Each player may sell at most 1 good to the trading house. The trading house buys only different goods up to a maximum of 4 goods in 1 phase. \textbf{Privilege:} The trader earns 1 extra for his sell. At the end of the phase, the Trader empties the trading house if the 4 slots are full.

\subsection*{The Captain}
Each player \emph{must} load goods on the cargo ships. This continues in player order as long as at least one player has goods he can load. Each ship only carries one type of good. Multiple ships can't carry the same type of good. On a player's turn, he may only load 1 type of good. The player must load as many goods as possible if there is space on the ship. For each good loaded, a player earns 1 victory point. \textbf{Privilege:} The captain earns 1 extra victory point. When no more goods can be loaded, players must store their remaining goods. Each player may store 1 good without a building. All other goods must be stored in one of his warehouses. Extra goods that can't be stored are returned to the supply. After the phase, any full ships are emptied.

\subsection*{The Prospectors}
No action is taken. \textbf{Privilege:} Take 1 doubloon from the bank.

\section*{End Game}
The game ends \emph{at the end} of the round in which one of the following occurs:
\begin{itemizeCustom}
	\item At the end of the Mayor phase there are not enough colonists to fill the colonist ship.
	\item During the Builder phase a player builds on his 12th building space.
	\item During the Captain phase the last of the victory point chips is used. In this case, players who should earn victory points but don't have chips to take should track them some other way.
\end{itemizeCustom}

The player adds his VP chips + the VP of his buildings + the extra VP of his large buildings (if they're occupied) to get his final score. The player with the most VPs wins! If there is a tie, the player with the most doubloons + goods is the winner.

\section*{Buildings}
Each player may only build each building \emph{once}. A building may be moved within the city to make room for a large building. The number of circles on a production building indicates the maximum number of goods the building can produce when the circles are occupied by colonists. The player must also have sufficient occupied plantations of the correct type. Check page 8 of the rules for an example of production.

\end{multicols*}
\end{document}
