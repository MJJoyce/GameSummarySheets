\documentclass[12pt]{article}

% Set landscape mode and custom margins for page while including space for the footer
\usepackage[includefoot, margin={0.5cm, 0.5cm}, landscape]{geometry}

% Used to reduce the spacing around section headers and titles.
% Format title headers by adjusting font size
% Set the spacing on all sides of the title to 0
\usepackage[compact]{titlesec}
\titleformat{\section}{\normalfont\bfseries}{\thesection}{1em}{}
\titlespacing*{\section}{0pt}{*0}{0pt}

\titleformat{\subsection}{\normalfont\bfseries}{\thesection}{0.3em}{}
\titlespacing*{\subsection}{0pt}{*0}{0pt}

% Reduce spacing around list items
\usepackage{enumitem}
\setlist{nolistsep}

% Set line spacing
\usepackage{setspace}
\singlespacing

% Create a multicolumn layout
% Set the amount of separation between columns
% Draw a vertical rule between the columns
\usepackage{multicol}
\setlength{\columnsep}{1cm}
\setlength{\columnseprule}{0.1pt}

% Add a frame around the content
\usepackage{mdframed}

% Custom footer (and header if I wanted)
\usepackage{fancyhdr}
\pagestyle{fancy}

% Set custom headers and footers for fancyhdr
\fancyhead{}
\fancyfoot[C]{Le Havre Summary Sheet v1.0}
\renewcommand{\headrulewidth}{0pt} 	% Remove horizontal rule from header
\renewcommand{\footrulewidth}{0pt}  % Remove horizontal rule from footer

% Custom frame style for mdframed
% The negative margin is needed to fix a weird spacing that I couldn't figure out
\mdfdefinestyle{customFrame}{%
    outerlinewidth = 0.4pt,
    innertopmargin = -0.3cm}

% Reduce whitespace in the enumerate list environment
\newenvironment{enumerateCustom}
{\begin{enumerate}
  \setlength{\itemsep}{1pt}
  \setlength{\parskip}{0pt}
  \setlength{\parsep}{0pt}}
{\end{enumerate}}

% Reduce whitespace in the itemize list environment
\newenvironment{itemizeCustom}
{\begin{itemize}
  \setlength{\itemsep}{1pt}
  \setlength{\parskip}{0pt}
  \setlength{\parsep}{0pt}}
{\end{itemize}}

\begin{document}
\begin{multicols*}{2}

\section*{Setup}
\begin{enumerateCustom}
	\item Set out the 3 game boards (Use numbers in top left corner for ordering).
	\item Give each player 1 Person disc, 1 Ship marker, and 1 Game turn overview card.
	\item Shuffle the 7 supply disks and place them face-down in the water spaces. Place players' ships beside the first supply tile.
	\item Place out the goods and food supply tokens.
	\item Place the starting goods (Check text on offer spaces for amount).
	\item Give each player 5 Francs and 1 coal.
	\item Set 6 Special Building cards face-down onto the marked space.
	\item Set out the starting building cards (Construction Firm and 2 Building Firms).
	\item Set out the Standard Buildings. Only use a card that has a dark tick (Or light if playing the short version) for the current number of players. Divide the cards into 3 piles and sort them into descending order according to the Sort Number (top right).
	\item Sort the Round cards for the appropriate number of players according to the dark circled number (or light if playing the short version).
	\item Set out the Loan Cards, Round Overview card, and Food Production tokens. Pick a starting player.
\end{enumerateCustom}

\section*{Goal of the Game}
Be the player with the largest fortune by constructing new buildings and ships all while making sure to keep your crew well fed. 

\section*{Player Turn}
A player's turn consists of two Mandatory actions as well as optional Additional actions. 

\subsection*{Supply Action}
At the start of his turn, a player first moves his ship to the next available Supply tile and distributes the marked goods to the corresponding Offer area. If the player moved his ship onto the Supply Tile with ``Interest'' on it, then all players with a loan must pay 1 Franc interest.

\subsection*{Main Action}
The Main Action is mandatory. The player may do one of the following:
\begin{itemizeCustom}
	\item Take all the goods or Francs from one of the Supply Spaces.
	\item Use a building. Move your player disk to an \emph{unoccupied} building. The player may not reuse the building where their Player Disk is currently placed. Only buildings that are owned by the town or a player may be used. Before taking the action, the Entry Fee (top right) must be paid. Food fees may be paid with food or Francs. Franc fees may only be paid with Francs.
\end{itemizeCustom}

\subsection*{Additional Actions}
\begin{itemizeCustom}
	\item \textbf{Buying:} At \emph{any} point during his turn, a player may buy one or more buildings and/or ship cards.
		\begin{itemizeCustom}
			\item Any building owned by the town and the buildings on the top of the three Proposal tiles may be purchased. If a separate cost is not listed, then the value is the price.
			\item The topmost face-up card on each ship pile may be purchased.
		\end{itemizeCustom}
	\item \textbf{Selling:} Buildings and Ships may be sold \emph{to the town} for \textbf{half} their value (Top left Franc number). Selling may happen on other players' turns, but not while they're taking an action. Buildings may not be sold and then bought again in the same game turn.
\end{itemizeCustom}

\section*{End of a Round}
After the end of every seventh game turn, the top Round card is resolved. The number in the pot symbol is the amount of food that each player must play during the Feeding Phase.
\begin{itemizeCustom}
	\item \textbf{Harvest:} If a player has at least 1 grain, they get another grain. If they have at least 2 cattle, they get another cattle.
	\item \textbf{Feeding Phase:} Each player must pay the food on this Round cards pot symbol. A player's ship(s) reduce the amount of food needed to be paid. If a player doesn't have enough food must either sell buildings or take a Loan card.
	\item \textbf{Town Construction:} If there is a standard or special building card symbol on the Round card then the town builds a new building. If it's a standard building, take the card with the lowest Sort Order number from the Building Proposals. If it's a special building, take the top Special Building from the supply.
	\item \textbf{New Ship:} Finally, flip over the Round card and place the ship on the appropriate pile.
\end{itemizeCustom}

\section*{Final Stage}
After the final Round card is resolved, the Final stage begins. Each player gets one more turn and carries out a final Main action. Supply and Buying actions are not allowed. No interest is paid. Players may use \textbf{occupied} buildings during the Final Stage. However, a player may not reuse the building which their disk is already on.

\section*{Final Scoring}
A player's wealth is calculated by adding:
\begin{itemizeCustom}
	\item The indicated value of their buildings and ships (Top Left Franc number).
	\item Additional value of buildings with a ``+'' symbol (e.g. the Bank).
	\item Any Francs.
	\item Deduct 7 Francs for each unpaid loan.
\end{itemizeCustom}

\end{multicols*}
\end{document}
