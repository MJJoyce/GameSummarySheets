\documentclass[12pt]{article}

% Set landscape mode and custom margins for page while including space for the footer
\usepackage[includefoot, margin={0.5cm, 0.5cm}, landscape]{geometry}

% Used to reduce the spacing around section headers and titles.
% Format title headers by adjusting font size
% Set the spacing on all sides of the title to 0
\usepackage[compact]{titlesec}
\titleformat{\section}{\large\bfseries}{\thesection}{1em}{}
\titleformat{\subsection}{\normalsize\bfseries}{\thesubsection}{1em}{}
\titleformat{\subsubsection}{\small\bfseries}{\thesubsubsection}{1em}{}
\titlespacing*{\section}{0pt}{*0}{0pt}
\titlespacing*{\subsection}{0pt}{*0}{0pt}
\titlespacing*{\subsubsection}{0pt}{*0}{0pt}

% Reduce spacing around list items
\usepackage{enumitem}
\setlist{nolistsep}

% Set line spacing
\usepackage{setspace}
\singlespacing

% Create a multicolumn layout
% Set the amount of separation between columns
% Draw a vertical rule between the columns
\usepackage{multicol}
\setlength{\columnsep}{1cm}
\setlength{\columnseprule}{0.1pt}

% Add a frame around the content
\usepackage{mdframed}

% Remove paragraph indent
\setlength{\parindent}{0pt}

% Custom footer (and header if I wanted)
\usepackage{fancyhdr}
\pagestyle{fancy}

% Set custom headers and footers for fancyhdr
\fancyhead{}
\fancyfoot[C]{Scythe Summary Sheet v1.0}
\renewcommand{\headrulewidth}{0pt} 	% Remove horizontal rule from header
\renewcommand{\footrulewidth}{0pt}  % Remove horizontal rule from footer

% Custom frame style for mdframed
% The negative margin is needed to fix a weird spacing that I couldn't figure out
\mdfdefinestyle{customFrame}{%
    outerlinewidth = 0.4pt,
    innertopmargin = -0.3cm}

% Reduce whitespace in the enumerate list environment
\newenvironment{enumerateCustom}
{\begin{enumerate}
  \setlength{\itemsep}{1pt}
  \setlength{\parskip}{0pt}
  \setlength{\parsep}{0pt}}
{\end{enumerate}}

% Reduce whitespace in the itemize list environment
\newenvironment{itemizeCustom}
{\begin{itemize}
  \setlength{\itemsep}{1pt}
  \setlength{\parskip}{0pt}
  \setlength{\parsep}{0pt}}
{\end{itemize}}

\begin{document}
%\begin{mdframed}[style = customFrame]
\begin{multicols*}{2}

\section*{Setup}
\begin{enumerateCustom}
    \item Place 1 \textbf{Encounter Token (green compass)} on each territory marked with encounter symbol.
    \item Place resource tokens, coins, and multiplier tokens next to board.
    \item Shuffle \textbf{Combat Cards (yellow)} and place on board.
    \item Shuffle Factor Cards (purple) and draw Player Count + 1. Place face-down on board and return remainder to box w/o looking.
    \item Shuffle Encounter Cards (green) and place on board.
    \item Shuffle Objective Cards (red) and place on board.
    \item Randomly select one structure bonus tile and place face-up below popularity track.
    \item Randomize Faction and Player Mats and deal 1 of each to players.
    \item Faction Mat specifies starting Power Track location and number of Combat Cards to draw.
    \item Player Mat specifies starting Popularity Track location, number of Objective Cards to draw, and starting coins.
    \item Place character on faction's home base. Place 1 workers on territories connected to home base.
    \item Place 6 technology cubes on Player Mat in green boxes w/ black square in corner.
    \item Place 6 remaining workers (meeples) on rectangles above Produce action.
    \item Place 4 structure tokens in corresponding boxes.
    \item Place 4 recruit tokens (cylinders) on circular bottom row spaces.
    \item Place 4 mechs on spaces on Faction Mat.
    \item Place 6 star tokens in upper left corner of Faction Mat.
    \item First Player is determined by number in the top right corner of Player Mat starting items.
\end{enumerateCustom}

\section*{Goal of the Game}
Make your faction the richest and most powerful in Eastern Europa by exploring and conquering territory, enlisting new recruits, producing resources and workers, build structures, and deploy mechs.

\section*{Player Turn}
Players take turns on after another until a player places their 6th star on the board. On your turn, do the following sequentially:

\begin{enumerateCustom}
    \item Place action token on a different section of Player Mat than previous turn.
    \item (Optionally) Take the top-row action in that section.
    \item (Optionally) Take the bottom-row action in that section.
\end{enumerateCustom}

    The costs (red boxes) and befits (green boxes) on Player Mat are the empty spaces \textbf{before} the action is taken. Resources gained from the top-row action may be used to pay for the bottom-row action. The name of an action is marked next to the benefits for taking the action.

\section*{Top-Row Actions}
\subsection*{Move}
\textbf{Either} move up to 2 different units you control from one territory to adjacent one or gain \$1

\begin{itemizeCustom}
    \item \textbf{Resources and Workers:} Units may pick/place any number of resource tokens during Move. Mechs can carry any number of resources and workers. Mech moving worker doesn't cost worker its move.
    \item \textbf{Rivers and Lakes:}  By default, units can't move across rivers or onto lakes. \textbf{River} is a body of water on the border between two land territories. \textbf{Lake} is a body of water comprising entire territory hex.
    \item \textbf{Tunnels:} All territories with the tunnel icon (red/black) are considered adjacent for Move.
    \item \textbf{Home Base:} By default, you can't Move any unit into a Home Base.
    \item \textbf{No Limit:} No limit to number of samee-faction units that share a territory.
    \item \textbf{Encounters:} If character moves into territory with encounter token, end movement and can't move again this turn. After all combats resolved, if character is still in such territory, discard encounter token and resolve.
    \item \textbf{Moving into opponent-controlled territories}
        \begin{itemizeCustom}
            \item Workers can't move by themselves into territory controlled by enemy units.
            \item \textbf{Controlled by workers:} End movement, can't move again this turn. After Move completed, all enemy workers retreat to home base. You lose 1 popularity per worker. 
            \item \textbf{Controlled by structure:} Any unit can move into territory controlled by only structure. Unit's player now controls territory.
            \item \textbf{Controlled by character/mech:} End movement. Opponent still temporarily controls territory. After Move completed, if any of your mechs/character share territory with opponent's, combat happens.
        \end{itemizeCustom}
\end{itemizeCustom}

\subsection*{Bolster}
Pay \$1 and gain either power on Power Track or draw Combat Card(s)

\subsection*{Produce}
Pay cost on all exposed red rectangle, choose different territories and each worker there produces 1 resource. Resource type is dictated by terrain type (Terrain Icon shows resource type). If a worker is produced, remove left-most worker from Player Mat \textbf{after} paying cost.

\subsection*{Trade}
Pay \$1 and either gain 2 resource tokens (any combination) and place on territory you control w/ at least 1 worker on it. Can't do this if all workers on Home Base. Or, increase Popularity on track.

\section*{Bottom-Row Actions}
\subsection*{Upgrade}
Pay cost (in oil resources), move technology cube from any green box to any empty red box \textbf{with bracketed borders}.

\subsection*{Deploy}
Pay cost (in metal resources), choose any mech on Faction Mat, place on territory you control w/ $\geq$ 1 worker. All mechs gain the ability lists on Faction Mat that was under mech.

\subsection*{Build}
Pay cost (in wood resources), take any structure from Player Mat, place on territory you control w/ $\geq$ 1 worker.

\begin{itemizeCustom}
    \item Only 1 structure per territory
    \item You may build on the Factory territory
    \item Can't build on lakes or Home Base
\end{itemizeCustom}


\subsection*{Enlist}
Pay cost (in food resources), take recruit token from any section of Player Mat, place in any open Recruit One-Time bonus space, gain depicted bonus.

\subsubsection*{Ongoing Bonus}
Each recruit gives a Recruit Ongoing Bonus related to the action the token was taken from. For remainder of game, when you or player to immediate left/right take bottom row action in same section you gain the specified bonus. Top row and similar Factory Card actions don't count. In 2 player game, you only gain the bonus once.

\section*{Combat}
Combat happens after Move before bottom-row action. Active player chooses order of combats. Attacking player resolves mech abilities affecting combat first.

\begin{enumerateCustom}
    \item \textbf{Select Power:} Simultaneously and secretly select power to spend on Power Dial. You must have the power to spend.
    \item \textbf{Combat Cards (optional):} For each character/mech in current combat, you may tuck 1 Combat Card from hand behind dial.
    \item \textbf{Reveal:} Reveal dials and cards. Add up total. Highest wins, tie goes to attacker. Deduct power cost on Dial from Track. Discard Combat Cards.
\end{enumerateCustom}

\subsection*{Winner}
\begin{enumerateCustom}
    \item Controls territory and all resources
    \item Place 1 start token on combat space of Triumph Track (if below max 2)
    \item If attacker won, lose 1 popularity for each worker forced to retreat
    \item If Encounter Token is on territory and character is there, resolve encounter
\end{enumerateCustom}

\subsection*{Loser}
\begin{enumerateCustom}
    \item Retreat all units to home base
    \item All carried resources are left on territory
    \item If loser revealed at least 1 power on dial or via Combat Card, draw 1 Combat Card.
\end{enumerateCustom}

\section*{Encounters}
If character moves into territory with encounter token, end movement and can't move again this turn. After all combats resolved, if character is still in such territory, discard encounter token and resolve.

\begin{enumerateCustom}
    \item Show card and read bold text aloud
    \item Pick one option (must be able to pay cost) and resolve
\end{enumerateCustom}

\section*{The Factory}
Once per game, if character is on Factory for 1st time, look throgh all Factory cards and choose one. Factory cards provide a fifth section of Player Mat to take actions with. All Factory Cards have a bottom-row move action. This says "move 1 unit up to 2 times". Other rules are consistent with this Move. May move through Mines with this. If unlocked Speed mech ability, move up to 3 spaces.

\section*{Objectives}
Completed objective cards may be revealed at any time during your turn. Place 1 star token on Triumph Track and discard object cards to bottom of objective deck.

\section*{End Game / Scoring}
The game ends \textbf{immediately} when a player places their 6th star on the board. This may interrupt certain actions.

\begin{enumerateCustom}
    \item If combats are left to resolve any units active player moved to initiate that combat are moved back.
    \item If Recruit Ongoing Bonuses are left to be resolved they are skipped.
    \item See Game End and Scoring for more Edge Cases.
\end{enumerateCustom}

The person with the most coins wins:
\begin{itemizeCustom}
    \item All accumulated coins
    \item Popularity Track \# coins per star on Triumph Track
    \item Popularity Track \# coins per controlled territory. Factory counts as 3 territories.
    \item Popularity Track \# coins per every 2 resources.
    \item Structure bonuses achieved on Structure Bonus Tile.
\end{itemizeCustom}

\section*{Misc.}
\textbf{Territory:} A territory is a hex on the board labeled with one of the colored hex terrain type icons.

\textbf{Home Base:} A home base is not a territory. Can't move units or build structures onto a home base.

\textbf{Control:} You control a territory if you have at least one unit there or if you have a structure there w/ no enemy units there.

\textbf{Resources:} Resource tokens remain on the board once produced. You may spend resource from any territories you control. Workers do not count as resources.

\textbf{Gaining Benefits:} When gaining a benefit from taking an action you may take only part of the benefit if desired.

\textbf{Alliance/Bribes:} Players may make informal agreements. Only tangible items exchangeable are coins. Can't negotiate out of combat that has already begun. Agreements aren't enforceable.

\subsection*{Types of Units}
3 Unit types (characters, mechs, and workers) can all move around the board (not over rivers/lakes by default) and transport any number of resource tokens. Plastic units can engage in combat, wood ones cannot.

\textbf{Characters} can engage in combat, have encounters, and (once a game) gain a Factory card

\textbf{Mechs} can engage in combat and transport any number of workers {\textbf{not characters}} when moving.

\textbf{Workers} can produce resource and workers, deploy mechs, and build structures.

\end{multicols*}
%\end{mdframed}
\end{document}
