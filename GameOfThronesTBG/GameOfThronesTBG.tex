\documentclass[10pt]{article}

% Set landscape mode and custom margins for page while including space for the footer
\usepackage[includefoot, margin={0.5cm, 0.5cm}, landscape]{geometry}

% Used to reduce the spacing around section headers and titles.
% Format title headers by adjusting font size
% Set the spacing on all sides of the title to 0
\usepackage[compact]{titlesec}
\titleformat{\section}{\normalfont\bfseries}{\thesection}{1em}{}
\titleformat{\subsection}{\normalfont\bfseries}{\thesection}{0.5em}{}
\titlespacing*{\section}{0pt}{*0}{0pt}
\titlespacing*{\subsection}{0pt}{*0}{0pt}

% Reduce spacing around list items
\usepackage{enumitem}
\setlist{nolistsep}

% Set line spacing
\usepackage{setspace}
\singlespacing

% Create a multicolumn layout
% Set the amount of separation between columns
% Draw a vertical rule between the columns
\usepackage{multicol}
\setlength{\columnsep}{1cm}
\setlength{\columnseprule}{0.1pt}

% Add a frame around the content
\usepackage{mdframed}

% Custom footer (and header if I wanted)
\usepackage{fancyhdr}
\pagestyle{fancy}

% Set custom headers and footers for fancyhdr
\fancyhead{}
\fancyfoot[C]{A Game of Thrones: The Board Game Summary Sheet v1.0}
\renewcommand{\headrulewidth}{0pt} 	% Remove horizontal rule from header
\renewcommand{\footrulewidth}{0pt}  % Remove horizontal rule from footer

% Custom frame style for mdframed
% The negative margin is needed to fix a weird spacing that I couldn't figure out
\mdfdefinestyle{customFrame}{%
    outerlinewidth = 0.4pt,
    innertopmargin = -0.3cm}

% Reduce whitespace in the enumerate list environment
\newenvironment{enumerateCustom}
{\begin{enumerate}
  \setlength{\itemsep}{1pt}
  \setlength{\parskip}{0pt}
  \setlength{\parsep}{0pt}}
{\end{enumerate}}

% Reduce whitespace in the itemize list environment
\newenvironment{itemizeCustom}
{\begin{itemize}
  \setlength{\itemsep}{1pt}
  \setlength{\parskip}{0pt}
  \setlength{\parsep}{0pt}}
{\end{itemize}}

\begin{document}
%\begin{mdframed}[style = customFrame]
\begin{multicols*}{2}

\section*{Setup}
\begin{enumerateCustom}
	\item Shuffle Wilding Deck (Mammoth Back) and place at top of board. Put \textbf{Wildling Threat} token on ``2''.
	\item Divide \textbf{Westeros} cards (Roman Numeral Back) into 3 decks. Shuffle and place near board.
	\item Find \textbf{Neutral Force} tokens based on player range. Place on board.
	% TODO Determine which houses based on players
	\item Determine player decks based on player numbers and distribute pieces.
	\item Place \textbf{Game Round} marker.
	\item place \textbf{influence}, \textbf{victory}, and \textbf{Supply} markers on board. Check player boards for positions of tokens.
	\item Place starting units and \textbf{Garrison} tokens.
	\item Place all \textbf{Power} tokens in a central pile. Each player receives 5 of their tokens.
\end{enumerateCustom}

\section*{Goal of the Game}
The game ends after 10 game rounds or if any player takes control of 7 areas containing a Castle or Stronghold. Player highest on victory track wins after 10 rounds.

\section*{Game Round}
\begin{itemizeCustom}
	\item \textbf{Westeros Phase:} Draw card from each deck. Resolve effects in order (I, II, III). Skip phase first game round.
	\item \textbf{Planning Phase:} Players simultaneously assign Order tokens to areas containing one or more of their units.
	\item \textbf{Action Phase:} Resolve order tokens.
\end{itemizeCustom}

\section*{Westeros Phase}
\begin{enumerateCustom}
	\item Advance game round marker
	\item Draw Westeros Cards
	\item Advance Wildling token for each Wildling icon drawn 
	\begin{itemizeCustom}
		\item If Wildling Thread reaches 12 immediately resolve Wildling Attack (See Wildling Attack)
	\end{itemizeCustom}
	\item Resolve Westeros Cards (See Westeros Cards)
\end{enumerateCustom}

\section*{Planning Phase}
\begin{enumerateCustom}
	\item Each player must assign one Order token facedown to each area that contains one of their units. Number of special order tokens cannot excede number of stars next to position on King's Court Influence track. If not enough tokens, place orders in turn order (not simulatenously). Player with too few can leave areas without orders.
	\item Reveal all order tokens simultaneously
	\item Player with Messenger Raven \emph{may} replace one of his order tokens for an unused one or look at top of Wildling deck. If look at deck, may put card back on top or bottom. May tell players what was on card, but can't show the card.
\end{enumerateCustom}

\section*{Action Phase}
\begin{enumerateCustom}
	\item Resolve Raid orders in turn order. Player \emph{may} remove adjacent enemy order. If remove Consolidate Power, gain one Power token and enemy loses one if possible.
	\item Resolve March orders and combat in turn order. 
	\begin{itemizeCustom}
		\item Player \emph{may} move units assigned order seperately or together into one or more adjacent areas
		\item Land units can't move to ports or sea areas
		\item Ship transportation allowed (See Ships)
		\item Can only move units into one area containing enemy units
		\item Units move into enemy controlled area starts combat after all units moved for that Movement order (See Combat)
		\item Player can leave no units in area. Loses control unless \textbf{Establishes Control}
	\end{itemizeCustom}
	\item Resolve Consolidate Power Orders in turn order. Player recieves one Power token + 1 per Power Icon on area.
	\item Clean up all remaining Orders. Routed units stood up. Refresh Messenger Raven and Valyrian Blade.
\end{enumerateCustom}

\begin{center}
\line(1,0){200}
\end{center}

\section*{Wildling Attack}
Initiated by Wildling Threat reaching 12 or Wildling Attack Westeros card being drawn and resolved. Two attacks in one phase are possible due to this.
\begin{enumerateCustom}
	\item \textbf{Determine Wildling Strength} from threat token positon.
	\item \textbf{Bid Power} tokens. Each player pick Power token bid secretly behind screen and puts in closed fist.
	\item \textbf{Calculate Night's Watch Strength} by totaling Power bid from all players.
	\item \textbf{Determine Outcome} by comparing Wilding Strength to Night's Watch Strength. Night's Watch win if greater or equal to Wildling strength.
	\item \textbf{Adjust Wildling Track} based on victor. Set to 0 if Night's Watch win. Move back 2 if Wildlings win.
	\item \textbf{Discard Power} tokens into main supply.
	\item \textbf{Resolve Wildling Attack Effects} based on top Wildling Deck card and outcome of battle. Place resolved card on bottom of deck. Bidding related ties are resolved by Iron Throne track holder.
\end{enumerateCustom}

\section*{Westeros Cards}
\begin{itemizeCustom}
	\item \textbf{Supply:} Each house, in turn order, counts number of Supply icons in areas it controls. Number of flags indicates number of armies and the number of units in each army. If current armies become invalid, remove units from board until supply limit is met.
	\item \textbf{Mustering:} Players, in turn order, may add units to areas containing Castle/Stronghold. Castles provide 1 point of muster and Strongholds 2. Footman and Ships cost 1 point of muster. Knights and Siege Engines cost 2 (1 if upgraded from Footman). Footman must be located in mustering area to be upgraded. Unit can't be mustered if it violates supply limit. Ships may be mustered into ports or adjacent sea area. Can't be put into sea area if enemy occupies it.
	\item \textbf{Clash of Kings:} Players bid for control of the influence tracks. Order is Iron Throne, Fiefdom, and then King's Court. All bidding is blind and bids are revealed simultaneously. Player with Iron Throne token breaks ties. All power tokens are returned to main supply. Highest bidder gets first location, second highest gets next location, etc.
\end{itemizeCustom}

\section*{Ships}
Any two land areas are considered adjacent for the purposes of marching and retreating if they're connected by one or more sea areas containing one or more friendly ships. Ships from other houses may never be used. Order token assigned to the ship area doesn't matter. Routed ships may be used. Ships can't move using ship transport. Units may use ship transport to move into enemy areas and start combat. Land areas are only considered adjacent for the purpose of Marching or Retreating, not for Supporting and Raiding.

\section*{Combat Steps}
\begin{enumerateCustom}
	\item \textbf{Call for Support} from area adjacent to combat containing Support Order. Support is optional. Player may support own units. If multiple areas can offer support, resolve in turn order. Support Order is not removed after combat. Siege Engines may only support attacks on Castle/Stronghold. Ships may support adjacent land, but land may not support adjacent sea.
	\item \textbf{Calculate Initial Combat Strength} for both sides. Footman/Ship are one. Knight is two. Siege Engine is four if against Castle/Stronghold. Add Defense/March Order bonuses to appropriate side. Add Supporting units and Special Order bonuses. Add Garrison defense strength.
	\item \textbf{Choose and Reveal House Cards} for both sides and resolve effects. Sword icon - enemy unit must be destroyed if enemy is defeated. Fortification icon - ignore one enemy Sword icon. Cards are discarded after. Last card is still discarded, remainder are put back in hand.
	\item \textbf{Use Valyrian Steel Blade} if either player has it and wants to.
	\item \textbf{Calculate Final Combat Strength} from initial combat strength + House Cards + Valyrian Steel Blade.
	\item \textbf{Combat Resolution}
	\begin{enumerateCustom}
		\item Victor is player with higher Final Combat Strength. Break ties with position on Fiefdoms track.
		\item Defeated player takes casualties equal to number of enemy sword icons minus fortification icons. Defender picks units to destroy. All types count as single ``unit'' regardless of strength. Support units cannot be destroyed from this. Siege units cannot be picked as the unit to destroy.
		\item Defeated units must retreat from area. All units to same area. Move to adjacent friendly or empty area (Can't be where initial march was from). Supply matters! Can't move to region if that violates Supply limit. If this is only option, destroy units until Supply limit is valid. If no legal location, all units are destroyed. Ship transport is allowed. Siege Engines can't retreat (just destroyed).
		\item Remove March Order. If Defender lost, remove any remaining Order or Power token in area.
	\end{enumerateCustom}
\end{enumerateCustom}

\section*{Controlling Areas}
Remember, a retreating army cannot retreat to an area with an enemy Power token.
\subsection*{Establishing Control}
When a player removes all units from an area through a March Order, that area no longer provides benefits to the player. If the player wants to maintain control, he can leave a power token in the region to maintain control. The power token is only removed once an enemy takes control of its area. A lone power token will not cause combat nor does it aid the defender. If a player doesn't have a power token, he can't establish control.
\subsection*{Controlling Home Areas}
Printed House shield of a home area functions like a permanent Power token. Players can control enemy home areas by keeping friendly units there or by establishing control in the area. Place power token directly over printed enemy house shield when establishing control. When the area is vacated, control reverts back to the original house.

\section*{Ports}
\begin{itemizeCustom}
	\item Ports must be assigned an order, supply limits are still in effect, and have a limit cap of 3 ships per port.
	\item A player may muster ships into a port even if the adjacent sea area is controlled by enemy units.
	\item Ports can't be attacked by enemy ships. Claim area with units to claim port.
	\item Ship units in port may support adjacent sea but not adjacent land area. They also don't contribute defense strength to adjacent land area.
	\item Ships in port may raid adjacent sea area. Adjacent sea area may raid port. Land units can't raid port.
	\item Consolidate Power Order on port is thrown away if enemy units control adjacent sea area.
	\item When ``Game of Thrones'' Westeros card is resolved, player gets additional power token if port's adjacent sea area is free of enemies. 
\end{itemizeCustom}

\section*{Neutral Force Tokens}
When attacking neutral forces:
\begin{itemizeCustom}
	\item No house cards are played.
	\item Neutral forces can't receive support.
	\item Valyrian Steel Blade can't be used.
	\item Marching against a neutral force counts as a Marching Order's single attack.
\end{itemizeCustom}
If a neutral card has no strength, it can't be taken!

\section*{Garrisons}
Garrisons provide extra defense to the starting area of each House.
\begin{itemizeCustom}
	\item Value of garrison is added to the defenders initial Combat Strength.
	\item Combat still occurs if there are no units defending the garrison's location.
	\item Garrisons are removed from the board when their area is defeated.
\end{itemizeCustom}

\end{multicols*}
%\end{mdframed}
\end{document}
