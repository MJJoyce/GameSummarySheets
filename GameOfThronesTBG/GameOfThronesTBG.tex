\documentclass[12pt]{article}

% Set landscape mode and custom margins for page while including space for the footer
\usepackage[includefoot, margin={0.5cm, 0.5cm}, landscape]{geometry}

% Used to reduce the spacing around section headers and titles.
% Format title headers by adjusting font size
% Set the spacing on all sides of the title to 0
\usepackage[compact]{titlesec}
\titleformat{\section}{\normalfont\bfseries}{\thesection}{1em}{}
\titlespacing*{\section}{0pt}{*0}{0pt}

% Reduce spacing around list items
\usepackage{enumitem}
\setlist{nolistsep}

% Set line spacing
\usepackage{setspace}
\singlespacing

% Create a multicolumn layout
% Set the amount of separation between columns
% Draw a vertical rule between the columns
\usepackage{multicol}
\setlength{\columnsep}{1cm}
\setlength{\columnseprule}{0.1pt}

% Add a frame around the content
\usepackage{mdframed}

% Custom footer (and header if I wanted)
\usepackage{fancyhdr}
\pagestyle{fancy}

% Set custom headers and footers for fancyhdr
\fancyhead{}
\fancyfoot[C]{A Game of Thrones: The Board Game Summary Sheet v1.0}
\renewcommand{\headrulewidth}{0pt} 	% Remove horizontal rule from header
\renewcommand{\footrulewidth}{0pt}  % Remove horizontal rule from footer

% Custom frame style for mdframed
% The negative margin is needed to fix a weird spacing that I couldn't figure out
\mdfdefinestyle{customFrame}{%
    outerlinewidth = 0.4pt,
    innertopmargin = -0.3cm}

% Reduce whitespace in the enumerate list environment
\newenvironment{enumerateCustom}
{\begin{enumerate}
  \setlength{\itemsep}{1pt}
  \setlength{\parskip}{0pt}
  \setlength{\parsep}{0pt}}
{\end{enumerate}}

% Reduce whitespace in the itemize list environment
\newenvironment{itemizeCustom}
{\begin{itemize}
  \setlength{\itemsep}{1pt}
  \setlength{\parskip}{0pt}
  \setlength{\parsep}{0pt}}
{\end{itemize}}

\begin{document}
%\begin{mdframed}[style = customFrame]
\begin{multicols*}{2}

\section*{Setup}
\begin{enumerateCustom}
	\item Shuffle Wilding Deck (Mammoth Back) and place at top of board. Put \textbf{Wildling Threat} token on ``2''.
	\item Divide \textbf{Westeros} cards (Roman Numeral Back) into 3 decks. Shuffle and place near board.
	\item Find \textbf{Neutral Force} tokens based on player range. Place on board.
	% TODO Determine which houses based on players
	\item Determine player decks based on player numbers and distribute pieces.
	\item Place \textbf{Game Round} marker.
	\item place \textbf{influence}, \textbf{victory}, and \textbf{Supply} markers on board. Check player boards for positions of tokens.
	\item Place starting units and \textbf{Garrison} tokens.
	\item Place all \textbf{Power} tokens in a central pile. Each player receives 5 of their tokens.
\end{enumerateCustom}

\section*{Goal of the Game}
The game ends after 10 game rounds or if any player takes control of 7 areas containing a Castle or Stronghold. Player highest on victory track wins after 10 rounds.

\section*{Game Round}
\begin{itemizeCustom}
	\item \textbf{Westeros Phase:} Draw card from each deck. Resolve effects in order (I, II, III). Skip phase first game round.
	\item \textbf{Planning Phase:} Players simultaneously assign Order tokens to areas containing one or more of their units.
	\item \textbf{Action Phase:} Resolve order tokens.
\end{itemizeCustom}

\section*{Westeros Phase}
\begin{enumerateCustom}
	\item Advance game round marker
	\item Draw Westeros Cards
	\item Advance Wildling token for each Wildling icon drawn 
	\begin{itemizeCustom}
		\item If Wildling Thread reaches 12 immediately resolve Wildling Attack
	\end{itemizeCustom}
	\item Resolve Westeros Cards
\end{enumerateCustom}

\section*{Planning Phase}
\begin{enumerateCustom}
	\item Each player must assign one Order token facedown to each area that contains one of their units. Number of special order tokens cannot excede number of stars next to position on King's Court Influence track. If not enough tokens, place orders in turn order (not simulatenously). Player with too few can leave areas without orders.
	\item Reveal all order tokens simultaneously
	\item Player with Messenger Raven \emph{may} replace one of his order tokens for an unused one or look at top of Wildling deck. If look at deck, may put card back on top or bottom. May tell players what was on card, but can't show the card.
\end{enumerateCustom}

\section*{Action Phase}
\begin{enumerateCustom}
	\item Resolve Raid orders in turn order. Player \emph{may} remove adjacent enemy order. If remove Consolidate Power, gain one Power token and enemy loses one if possible.
	\item Resolve March orders and combat in turn order. 
	\begin{itemizeCustom}
		\item Player \emph{may} move units assigned order seperately or together into one or more adjacent areas
		\item Land units can't move to ports or sea areas
		\item Ship transportation allowed (See Ships)
		\item Can only move units into one area containing enemy units
		\item Units move into enemy controlled area starts combat after all units moved for that Movement order (See Combat)
		\item Player can leave no units in area. Loses control unless \textbf{Establishes Control}
	\end{itemizeCustom}
	\item Resolve Consolidate Power Orders in turn order. Player recieves one Power token + 1 per Power Icon on area.
	\item Clean up all remaining Orders. Routed units stood up. Refresh Messenger Raven and Valyrian Blade.
\end{enumerateCustom}

\section*{Wildling Attack}
\section*{Ships}
\section*{Combat}
\begin{enumerateCustom}
	\item \textbf{Call for Support} from area adjacent to combat containing Support Order. Support is optional. Player may support own units. If multiple areas that can offer support, resolve in turn order. Support Order is not removed after combat. Siege Engines may only support attacks on Castle/Stronghold. Ships may support adjacent land, but land may not support adjacent sea.
	\item \textbf{Calculate Inital Combat Strength} for both sides. Footman/Ship are one. Knight is two. Siege Engine is four if against Castle/Stronghold. Add Defense/March Order bonuses to appropriate side. Add Supporting units and Special Order bonuses. Add Garrison defense strength.
	\item \textbf{Choose and Reveal House Cards} for both sides and resolve effects. Sword icon - enemy unit must be destroyed if enemy is defeated. Fortification icon - ignore one enemy Sword icon. Cards are discarded after. Last card is still discarded, remainder are put back in hand.
	\item \textbf{Use Valyrian Steel Blade} if either player has it and wants to.
	\item \textbf{Calculate Final Combat Strength} from intial combat strength + House Cards + Valyrian Steel Blade.
	\item \textbf{Combat Resolution}
	\begin{enumerateCustom}
		\item Victor is player with higher Final Combat Strength. Break ties with position on Fiefdoms track.
		\item Defeated player takes casulties equal to number of enemy sword icons minus fortification icons. Defender picks units to destroy. All types count as single ``unit'' regardless of strength. Support units cannot be destroyed from this.
		\item Defeated units must retreat from area. All units to same area. Move to adjacent friendly or empty area (Can't be where inital march was from). Supply matters! Can't move to region if that violates Supply limit. If this is only option, destroy units until Supply limit is valid. If no legal location, all units are destroed. Ship transport is allowed. Siege Engines can't retreat (just destroyed).
		\item Remove March Order. If Defender lost, remove any remaining Order or Power token in area.
	\end{enumerateCustom}
\end{enumerateCustom}
\section*{Establishing Control}

\end{multicols*}
%\end{mdframed}
\end{document}
