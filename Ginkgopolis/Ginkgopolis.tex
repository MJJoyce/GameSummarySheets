\documentclass[12pt]{article}

% Set landscape mode and custom margins for page while including space for the footer
\usepackage[includefoot, margin={0.5cm, 0.5cm}, landscape]{geometry}

% Used to reduce the spacing around section headers and titles.
% Format title headers by adjusting font size
% Set the spacing on all sides of the title to 0
\usepackage[compact]{titlesec}
\titleformat{\section}{\normalfont\bfseries}{\thesection}{1em}{}
\titlespacing*{\section}{0pt}{*0}{0pt}

% Reduce spacing around list items
\usepackage{enumitem}
\setlist{nolistsep}

% Set line spacing
\usepackage{setspace}
\singlespacing

% Create a multicolumn layout
% Set the amount of separation between columns
% Draw a vertical rule between the columns
\usepackage{multicol}
\setlength{\columnsep}{1cm}
\setlength{\columnseprule}{0.1pt}

% Add a frame around the content
\usepackage{mdframed}

% Custom footer (and header if I wanted)
\usepackage{fancyhdr}
\pagestyle{fancy}

% Set custom headers and footers for fancyhdr
\fancyhead{}
\fancyfoot[C]{Ginkgopolis Summary Sheet v1.1}
\renewcommand{\headrulewidth}{0pt} 	% Remove horizontal rule from header
\renewcommand{\footrulewidth}{0pt}  % Remove horizontal rule from footer

% Custom frame style for mdframed
% The negative margin is needed to fix a weird spacing that I couldn't figure out
\mdfdefinestyle{customFrame}{%
    outerlinewidth = 0.4pt,
    innertopmargin = -0.3cm}

% Reduce whitespace in the enumerate list environment
\newenvironment{enumerateCustom}
{\begin{enumerate}
  \setlength{\itemsep}{1pt}
  \setlength{\parskip}{0pt}
  \setlength{\parsep}{0pt}}
{\end{enumerate}}

% Reduce whitespace in the itemize list environment
\newenvironment{itemizeCustom}
{\begin{itemize}
  \setlength{\itemsep}{1pt}
  \setlength{\parskip}{0pt}
  \setlength{\parsep}{0pt}}
{\end{itemize}}

\begin{document}
%\begin{mdframed}[style = customFrame]
\begin{multicols*}{2}

\section*{Setup}
\begin{enumerateCustom}
	\item Place the 9 \emph{Building} tiles numbered 1, 2, and 3 in a 3x3 square.
	\item Place the 12 \emph{Urbanization} tokens in alphabetical order around the board.
	\item Shuffle the remaining \emph{Building} tiles and place in face down piles. For 2 or 3 player game, remove 6 random tiles without peeking!
	\item Form a deck by shuffling the 12 \emph{Urbanization} cards (A to L) and the 9 \emph{Building} cards that correspond to the starting \emph{Building} tiles. In 2 or 3 player game, discard the first 7 cards face up next to the deck.
	\item Sort the \emph{Building} cards numbered 4-20 by color and ascending order.
	\item Each player picks a color and places the resources in the general supply. Take 25/20/18/16 resources for a 2/3/4/5 player game.
	\item Each player gets a screen in their color and 2 \emph{New Hand} tokens.
	\item Give each player 4 character cards. Players take 1 card from their hand and pass the remainder left. Continue until everyone has 3. Then everyone reveals their character cards. In introductory game, each player receives a random set of 3 cards that are identified by the same number.
	\item Place character cards in front of player screen. Take the resources indicated by the starting character cards in the top-left corner. All items go behind your screen.
	\item Deal 4 cards from the deck to each player.
\end{enumerateCustom}

\section*{Goal of the Game}
Players are trying to accrue the most ``success'' points by the end of the game by operating and constructing buildings in the city of Ginkgopolis.

\section*{Playing the Game}
The game is divided into 3 phases:
\begin{enumerateCustom}
	\item Choose a card
	\item Resolve actions
	\item Prepare for the next round
\end{enumerateCustom}

\section*{Choose a Card}
Players simultaneously look at their hand of 4 cards, pick one, and place it face down in front of their screen. The card can be played by itself or with a \emph{Building} tile, depending on the action that you wish to perform. If playing it with a tile, select the tile behind your screen and place it face down on the card. Players may also discard a \emph{New Hand} token to discard all 4 cards and draw 4 new cards.

\section*{Resolve Actions}
Starting with the first player, each player reveals their chosen card(and tile), then resolves the corresponding action. The 3 possible actions are:
\begin{itemizeCustom}
	\item Exploiting: Playing a card by itself
	\item Urbanization: Playing an Urbanization card with a tile
	\item Constructing a floor: Playing a Building card with a tile
\end{itemizeCustom}

\section*{Exploiting}
\begin{itemizeCustom}
	\item If played only a \emph{Urbanization} card, take either a resource or tile from general supply.
	\item If played only a \emph{Building} card, take resources designated by tile's color. Red gives resources. Blue gives tiles. Yellow gives ``success'' points. Number of items received is the height of the building.
	\item Get bonuses from cards with \emph{Exploiting} bonuses on the bottom. These are on character cards and cards earned through constructing a floor.
	\item Discard the card.
\end{itemizeCustom}

\section*{Urbanization}
\begin{itemizeCustom}
	\item Replace the corresponding \emph{Urbanization} token with the selected tile. Also place a \emph{Construction Site} pawn and a resource (from behind the player's screen) on the tile.
	\item Place \emph{Urbanization} token orthogonally adjacent to the new tile. If not possible, move other tokens so it can be placed. Tokens must remain in alphabetical order!
	\item Buildings orthogonally adjaced to the new tile are ``utilized'' by the player. Get bonuses as though you had exploited those tiles.
	\item Get bonuses from cards with an \emph{Urbanization} bonus.
	\item Discard the \emph{Urbanization} card.
\end{itemizeCustom}

\section*{Constructing a Floor}
Note, if you don't have the items to perform all these steps you must then Exploit the card and take back all your resources and tiles.
\begin{itemizeCustom}
	\item Return resources from the building being constructed to the corresponding player. If it it a different player, they get a success point for each resource.
	\item Place tile atop the building. If number on tile is less than the covered tile number, pay the difference in success points. If the tiles aren't the same color, discard a resource to general supply.
	\item Place resources equal to height of the building onto the tile. Place \emph{Construction Site} pawn onto tile.
	\item Get bonuses from cards with a \emph{Constructing a floor} bonus.
	\item Keep the card face up in your area. It gives bonuses for the rest of the game.
\end{itemizeCustom}

\section*{Prepare for the Next Round}
\begin{itemizeCustom}
	\item Take all unused cards (Including the First Player Card if present) from your right-hand neighbor.
	\item Draw up to a hand of 4 cards from the deck of unused cards starting with the new first player.
	\item If at any time the deck is exhausted, the first player must:
	\begin{enumerateCustom}
		\item For each tile with a \emph{Construction Site} pawn on it, find the corresponding card in one of the stacks of \emph{Building} cards.
		\item Shuffle the deck, discard pile, and new cards to form a new deck.
		\item If 2 or 3 players, discard the top 7 cards before dealing players new hands.
		\item Remove all \emph{Construction Site} pawns.
	\end{enumerateCustom}
\end{itemizeCustom}

\section*{End Game}
If the tile supply has been exhausted, players may add as many tiles from behind their screen to form a new supply. Players get 1 point for each tile they give.\\
The game ends after the current round when:
\begin{itemizeCustom}
	\item The tile supply has been exhausted a second time.
	\item A player has placed all their resources into the city.
\end{itemizeCustom}

\section*{Scoring}
Add up points as follows:
\begin{itemizeCustom}
	\item Any score tokens earned in the game.
	\item Cards with an endgame bonus (Will have an '=' sign).
	\item 2 points for each unused \emph{New Hand} token.
	\item For each district in the city (district is area formed by two or more adjacent tiles of same color):
		\begin{itemizeCustom}
			\item Player with highest number of resources in district gets points equal to total number of resources in district.
			\item Player with second highest gets points equal to his number of resources in district.
			\item If there's a tie, player with highest building wins. If still a tie, the number on the highest buildings is used to break the tie.
		\end{itemizeCustom}
\end{itemizeCustom}

\end{multicols*}
%\end{mdframed}
\end{document}
