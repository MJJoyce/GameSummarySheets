\documentclass[12pt]{article}

% Set landscape mode and custom margins for page while including space for the footer
\usepackage[includefoot, margin={0.5cm, 0.5cm}, landscape]{geometry}

% Used to reduce the spacing around section headers and titles.
% Format title headers by adjusting font size
% Set the spacing on all sides of the title to 0
\usepackage[compact]{titlesec}
\titleformat{\section}{\normalfont\bfseries}{\thesection}{1em}{}
\titlespacing*{\section}{0pt}{*0}{0pt}

% Reduce spacing around list items
\usepackage{enumitem}
\setlist{nolistsep}

% Set line spacing
\usepackage{setspace}
\singlespacing

% Create a multicolumn layout
% Set the amount of separation between columns
% Draw a vertical rule between the columns
\usepackage{multicol}
\setlength{\columnsep}{1cm}
\setlength{\columnseprule}{0.1pt}

% Add a frame around the content
\usepackage{mdframed}

% Custom footer (and header if I wanted)
\usepackage{fancyhdr}
\pagestyle{fancy}

% Set custom headers and footers for fancyhdr
\fancyhead{}
\fancyfoot[C]{Android Netrunner Summary Sheet v1.0}
\renewcommand{\headrulewidth}{0pt} 	% Remove horizontal rule from header
\renewcommand{\footrulewidth}{0pt}  % Remove horizontal rule from footer

% Custom frame style for mdframed
% The negative margin is needed to fix a weird spacing that I couldn't figure out
\mdfdefinestyle{customFrame}{%
    outerlinewidth = 0.4pt,
    innertopmargin = -0.3cm}

% Reduce whitespace in the enumerate list environment
\newenvironment{enumerateCustom}
{\begin{enumerate}
  \setlength{\itemsep}{1pt}
  \setlength{\parskip}{0pt}
  \setlength{\parsep}{0pt}}
{\end{enumerate}}

% Reduce whitespace in the itemize list environment
\newenvironment{itemizeCustom}
{\begin{itemize}
  \setlength{\itemsep}{1pt}
  \setlength{\parskip}{0pt}
  \setlength{\parsep}{0pt}}
{\end{itemize}}

% Prevent paragraph indent
\setlength{\parindent}{0pt}

\begin{document}
\begin{mdframed}[style = customFrame]
\begin{multicols*}{2}

\section*{Setup}
\begin{enumerateCustom}
	\item Pick sides, grab/create decks, and shuffle
	\item Set tokens out in convenient location near both players
	\item 5 \textbf{credits} to each player
	\item Draw 5 cards for starting hand. Players can \textbf{mulligan} by reshuffling their hand into deck and redraw-ing. Must keep second hand.
\end{enumerateCustom}

\section*{Goal of the Game}
\begin{itemizeCustom}
	\item Corporation wins if:
		\begin{itemizeCustom}
			\item Collect 7 \textbf{agenda} points from \textbf{Agenda} cards.
			\item The runner has hand size of 0 at end of runner's turn.
		\end{itemizeCustom}
	\item Runner wins if:
		\begin{itemizeCustom}
			\item Collect 7 \textbf{agenda} points from \textbf{Agenda} cards.
			\item Corporation has no card to draw from their \textbf{R\&D}.
		\end{itemizeCustom}
\end{itemizeCustom}

\section*{Player Turn}
The Corporation player begins the game and their turn has three phases.
\begin{enumerateCustom}
	\item \textbf{Draw Phase:} Draw a card from \textbf{R\&D}
	\item \textbf{Action Phase:} Perform 3 actions by spending \textbf{clicks}
	\item \textbf{Discard Phase:} Discard down to maximum hand size, if necessary
\end{enumerateCustom}

The Corporation player can perform any of these actions any number of times, assuming they can be paid for. Note, actions are taxen by spending \textbf{clicks}
\begin{enumerateCustom}
	\item Draw one card from \textbf{R\&D}
	\item Gain one \textbf{credit}
	\item Install an \textbf{agenda}, \textbf{asset}, \textbf{upgrade}, or piece of \textbf{ice}
	\item Play an \textbf{operation}
	\item Pay one \textbf{credit}: Advance a card
	\item Pay two \textbf{credits}: Trash a resource in Runner's rig if Runner is \textbf{TAGGED}
	\item Pay three \textbf{clicks}: Purge virus counters.
	\item Trigger a \textbf{click} ability on a card (cost varies).
\end{enumerateCustom}

The Runner player's turn has two phases.
\begin{enumerateCustom}
	\item \textbf{Action Phase:} Perform 4 actions by spending \textbf{clicks}
	\item \textbf{Discard Phase:} Discard down to maximum hand size, if necessary
\end{enumerateCustom}

The Runner player can perform any of these actions any number of times, assuming they can be paid for.
\begin{enumerateCustom}
	\item Draw one card from the \textbf{stack}
	\item Game one \textbf{credit}
	\item Install a \textbf{program}, \textbf{resource}, or piece of \textbf{hardware}
	\item Play an \textbf{event}
	\item Pay two \textbf{credits}: Remove one \textbf{tag}
	\item Make a run
	\item Trigger a \textbf{click} ability on a card (cost varies).
\end{enumerateCustom}
\section*{End Game}

\end{multicols*}
\end{mdframed}
\end{document}
