\documentclass[12pt]{article}

% Set landscape mode and custom margins for page while including space for the footer
\usepackage[includefoot, margin={0.5cm, 0.5cm}, landscape]{geometry}

% Used to reduce the spacing around section headers and titles.
% Format title headers by adjusting font size
% Set the spacing on all sides of the title to 0
\usepackage[compact]{titlesec}
\titleformat{\section}{\normalfont\bfseries}{\thesection}{1em}{}
\titlespacing*{\section}{0pt}{*0}{0pt}

% Reduce spacing around list items
\usepackage{enumitem}
\setlist{nolistsep}

% Set line spacing
\usepackage{setspace}
\singlespacing

% Create a multicolumn layout
% Set the amount of separation between columns
% Draw a vertical rule between the columns
\usepackage{multicol}
\setlength{\columnsep}{1cm}
\setlength{\columnseprule}{0.1pt}

% Add a frame around the content
\usepackage{mdframed}

% Custom footer (and header if I wanted)
\usepackage{fancyhdr}
\pagestyle{fancy}

% Set custom headers and footers for fancyhdr
\fancyhead{}
\fancyfoot[C]{Android Netrunner Summary Sheet v1.0}
\renewcommand{\headrulewidth}{0pt} 	% Remove horizontal rule from header
\renewcommand{\footrulewidth}{0pt}  % Remove horizontal rule from footer

% Custom frame style for mdframed
% The negative margin is needed to fix a weird spacing that I couldn't figure out
\mdfdefinestyle{customFrame}{%
    outerlinewidth = 0.4pt,
    innertopmargin = -0.3cm}

% Reduce whitespace in the enumerate list environment
\newenvironment{enumerateCustom}
{\begin{enumerate}
  \setlength{\itemsep}{1pt}
  \setlength{\parskip}{0pt}
  \setlength{\parsep}{0pt}}
{\end{enumerate}}

% Reduce whitespace in the itemize list environment
\newenvironment{itemizeCustom}
{\begin{itemize}
  \setlength{\itemsep}{1pt}
  \setlength{\parskip}{0pt}
  \setlength{\parsep}{0pt}}
{\end{itemize}}

% Prevent paragraph indent
\setlength{\parindent}{0pt}

\begin{document}
%\begin{mdframed}[style = customFrame]
\begin{multicols*}{2}

\section*{Setup}
\begin{enumerateCustom}
	\item Pick sides, grab/create decks, and shuffle
	\item Set tokens out in convenient location near both players
	\item 5 \textbf{credits} to each player
	\item Draw 5 cards for starting hand. Can mulligan \emph{once}.
\end{enumerateCustom}

\section*{Goal of the Game}
\begin{itemizeCustom}
	\item Corporation wins if:
		\begin{itemizeCustom}
			\item Collect 7 \textbf{agenda} points from \textbf{Agenda} cards.
			\item The runner has hand size of less than 0 at end of runner's turn.
			\item The runner takes more damage than the number of cards in his hand.
		\end{itemizeCustom}
	\item Runner wins if:
		\begin{itemizeCustom}
			\item Collect 7 \textbf{agenda} points from \textbf{Agenda} cards.
			\item Corporation has no card in \textbf{R\&D} and attempts to draw.
		\end{itemizeCustom}
\end{itemizeCustom}

\section*{Player Turn}
The Corporation player begins the game. Actions can be performed any number of times in any order.
\begin{enumerateCustom}
	\item \textbf{Draw Phase:} Draw a card from \textbf{R\&D}
	\item \textbf{Action Phase:} Perform 3 actions by spending \textbf{clicks}
	\item \textbf{Discard Phase:} Discard down to maximum hand size, if necessary
\end{enumerateCustom}
Possible actions:
\begin{enumerateCustom}
	\item Draw one card from \textbf{R\&D}
	\item Gain one \textbf{credit}
	\item Install an \textbf{agenda}, \textbf{asset}, \textbf{upgrade}, or piece of \textbf{ice}
	\item Play an \textbf{operation}
	\item Pay one \textbf{credit}: Advance a card
	\item Pay two \textbf{credits}: Trash a Runner's resource if Runner is \textbf{tagged}
	\item Pay three \textbf{clicks}: Purge virus counters.
	\item Trigger a \textbf{click} ability on a card (cost varies).
\end{enumerateCustom}

The Runner player's turn has two phases.
\begin{enumerateCustom}
	\item \textbf{Action Phase:} Perform 4 actions by spending \textbf{clicks}
	\item \textbf{Discard Phase:} Discard down to maximum hand size, if necessary
\end{enumerateCustom}
Possible actions:
\begin{enumerateCustom}
	\item Draw one card from the \textbf{stack}
	\item Gain one \textbf{credit}
	\item Install a \textbf{program}, \textbf{resource}, or piece of \textbf{hardware}
	\item Play an \textbf{event}
	\item Pay two \textbf{credits}: Remove one \textbf{tag}
	\item Make a run
	\item Trigger a \textbf{click} ability on a card (cost varies).
\end{enumerateCustom}

\section*{Action Explanations}
Corporation: Installing Cards
\begin{itemizeCustom}
	\item \textbf{Assets} and \textbf{upgrades} are played unrezzed. 
	\item When installing, Corp can first \textbf{trash} cards in that server. The trashed cards go to the \textbf{Archives} faceup if rezzed, facedown if unrezzed.
	\item A remote server can be created by installing there. If \textbf{ice} is used, it is \textbf{empty}. Can still be run against.
	\item \textbf{Agendas/Assets:} Can only be installed in remote server. Only one \textbf{agenda} or \textbf{asset} per remote server. \textbf{Upgrades} don't have to be trashed!
	\item \textbf{Upgrades:} Installed in server \textbf{root} when put into central server, otherwise put with agenda/asset. No install limit. Only one \textbf{region} subtype installed per \emph{server}.
	\item \textbf{Ice:} Installed in front of any server (sideways). Installed in outermost position. Must pay cost equal to number of \textbf{ice} in server. Already installed \textbf{ice} may be trashed to reduce install cost.
\end{itemizeCustom}

Corporation: Advancing a Card
\begin{itemizeCustom}
	\item One advancement token placed on installed card. \textbf{Agendas} can always be advanced, others if they state so.
	\item There is no limit to how many times a card can be advanced. 
	\item When \textbf{agenda} is advanced to its advancement requirement, it can be scored. Scoring does \emph{not} cost a \textbf{click}. Scoring is \emph{not} mandatory.
\end{itemizeCustom}

Runner: Installing Cards
\begin{itemizeCustom}
	\item \textbf{Programs:} Pay install cost and place it faceup into program row of \textbf{rig}. Programs have a \textbf{memory cost}. Runner starts with 4 \textbf{memory units (MUs)}. If installed programs memory cost is ever greater than the runners memory units, programs must be trashed.
	\item \textbf{Resources:} Pay install cost and place faceup into resource row of \textbf{rig}. No limit to number of installed resources.
	\item \textbf{Hardware:} Pay install cose and place faceup into hardware row of \textbf{rig}. No limit to number of installed hardware. Runner can only have one hardware with \textbf{console} subtype installed.
	% TODO: FAQ says that existing console can't be trashed I believe. Check this.
\end{itemizeCustom}

Runner: Runs
\begin{itemizeCustom}
	\item Primary interation is between Corporation's \textbf{Ice} and Runner's \textbf{Icebreakers}. Icebreakers can break the subroutines of ice so long as the two subtypes match and the icebreaker's level is greater than or equal to the ice's level. See run flow chart for more thorough timing explanations.
	\item \textbf{Initiation phase:} Runner declares server that the run is against. Receives 1 \textbf{credit} for each point of \textbf{bad publicity} on the Corporation. If there are ice protracting the server, proceed to \textbf{Confrontation phase}, otherwise go to \textbf{Access phase}.
	\item \textbf{Confrontaton phase:} Runner approaches ice starting from the outermost. When approached any piece of ice except the first of the run, the Runner may \textbf{Jack Out} and end the run. For each piece of ice, if the ice is rezzed then the runner encounters it. Otherwise, if the Corporation rezzes the card, the Runner enounters it, otherwise the Runner passes the ice. For an encountered piece of ice, the Runner has the opportunity to break subroutines on the ice in any order. Any unbroken subroutines then trigger and the ice is passed, assuming the run wasn't ended by a subroutine. If all the ice protecting server are passed, proceed to \textbf{Access phase}.
	\item \textbf{Access phase:} The Access phase is different depending on the server run against. Cards can be trashed by paying the trash cost (trash can icon) on the card.
	\begin{itemizeCustom}
		\item \textbf{R\&D:} Access top card and any upgrades in root. Cards aren't show to Corporation unless they are scored, trashed, or forced to reveal by card text.
		\item \textbf{HQ:} Access one random card and any upgrades in root. 
		\item \textbf{Archives:} Access all cards in \textbf{Archives} and any upgrades in root. All cards are turned faceup and order does not need to be maintained. All \textbf{agendas} are stolen, but cannot trash any cards already in the \textbf{Archives}.
		\item \textbf{Remote Server:} Access all cards in server.
	\end{itemizeCustom}
\end{itemizeCustom}

\section*{Additional Rules}
Traces:
\begin{itemizeCustom}
	\item Some card abilities start a trace. Trace$^{\textrm{x}}$ where \textbf{x} is the base strength.
	\item First, Corporation may spend \textbf{credits} to increase trace strength by one per \textbf{credit}.
	\item Next, Runner may increase base link strength by spending credits. Runner's base link strength is equal to the number of \textbf{Links} it has in play.
	\item Compare trace and link strengths. If trace is greater than link, trace is successful. Resolve any ``if successful'' effects associated with the trace. Otherwise, resolve ``if unsuccessful'' effects.
\end{itemizeCustom}

Tags
\begin{itemizeCustom}
	\item Some cards place a tag marker on the Runner. If Runner has at least one tag, it is \textbf{Tagged}. When tag, Corporation can trash resources as an action and the Runner can remove a tag as an action.
\end{itemizeCustom}

Damage
\begin{itemizeCustom}
	\item \textbf{Meat/Net Damage:} Differ only by name. Runner randomly trashes a card from \textbf{grip} for each such damage.
	\item \textbf{Brain Damage:} Runner randomly trashes one card from \textbf{grip} and has maximum hand size reduced by 1. Take brain damage token to track this.
\end{itemizeCustom}

Hosting
\begin{itemizeCustom}
	\item Some cards are installed on other cards (``hosted card'') and some allow cards to be installed on them (``host card''). Some cards can also host counters and tokens. If a trigger cost requires a hosted counter, the counters must be spent from the card the ability is on.
\end{itemizeCustom}
\end{multicols*}
%\end{mdframed}
\end{document}
