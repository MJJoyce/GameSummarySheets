\documentclass[12pt]{article}

% Set landscape mode and custom margins for page while including space for the footer
\usepackage[includefoot, margin={0.5cm, 0.5cm}, landscape]{geometry}

% Used to reduce the spacing around section headers and titles.
% Format title headers by adjusting font size
% Set the spacing on all sides of the title to 0
\usepackage[explicit]{titlesec}
\titleformat{\section}{\large\bfseries}{\thesection}{1em}{\underline{#1}}
\titleformat{\subsection}{\normalsize\bfseries}{\thesubsection}{1em}{#1}
\titleformat{\subsubsection}{\small\bfseries}{\thesubsubsection}{1em}{#1}
\titlespacing*{\section}{0pt}{*0}{0pt}
\titlespacing*{\subsection}{0pt}{*0}{0pt}
\titlespacing*{\subsubsection}{0pt}{*0}{0pt}

% Reduce spacing around list items
\usepackage{enumitem}
\setlist{nolistsep}

% Set line spacing
\usepackage{setspace}
\singlespacing

% Create a multicolumn layout
% Set the amount of separation between columns
% Draw a vertical rule between the columns
\usepackage{multicol}
\setlength{\columnsep}{1cm}
\setlength{\columnseprule}{0.1pt}

% Add a frame around the content
\usepackage{mdframed}

% Remove paragraph indent
\setlength{\parindent}{0pt}

% Custom footer (and header if I wanted)
\usepackage{fancyhdr}
\pagestyle{fancy}

% Set custom headers and footers for fancyhdr
\fancyhead{}
\fancyfoot[C]{Root Summary Sheet v1.0}
\renewcommand{\headrulewidth}{0pt} 	% Remove horizontal rule from header
\renewcommand{\footrulewidth}{0pt}  % Remove horizontal rule from footer

% Custom frame style for mdframed
% The negative margin is needed to fix a weird spacing that I couldn't figure out
\mdfdefinestyle{customFrame}{%
    outerlinewidth = 0.4pt,
    innertopmargin = -0.3cm}

% Reduce whitespace in the enumerate list environment
\newenvironment{enumerateCustom}
{\begin{enumerate}
  \setlength{\itemsep}{1pt}
  \setlength{\parskip}{0pt}
  \setlength{\parsep}{0pt}}
{\end{enumerate}}

% Reduce whitespace in the itemize list environment
\newenvironment{itemizeCustom}
{\begin{itemize}
  \setlength{\itemsep}{1pt}
  \setlength{\parskip}{0pt}
  \setlength{\parsep}{0pt}}
{\end{itemize}}

\begin{document}
\begin{multicols*}{2}

\section*{Setup}
\begin{enumerateCustom}
    \item Choose factions, take all faction's pieces, put score markers on 0.
    \item Shuffle cards and deal 3 to each player. Set combat dice by map.
    \item Place Ruin chits on slots marked "R" on map.
    \item Gather 12 item chits and place on spaces on top of map.
    \item Setup factions in alphabetical order listed in Setup section.
\end{enumerateCustom}

\section*{Goal of the Game}
Take over the woodlands by being the first Faction to 30 points or to complete a dominance card.

\section*{General Actions and Concepts}
\subsection*{Moving}
Move any number of warriors from one Clearing to path-connected Clearing. Must \textbf{Rule} Clearing moving from \textbf{or} to.

\subsection*{Rule}
Rule a clearing if you have most combined warriors + buildings there.

\subsection*{Clearings}
Each Clearing has 1 -- 3 slots where \textbf{Buildings} are placed and a Suit representing the community living there.

\subsection*{Ruins}
Ruins fill some slots in Clearings. Until a Vagabond clears a Ruin they can't hold buildings.

\subsection*{Suits}
There are 4 Suits: Fox, Mouse, Rabbit, and Bird. Birds are wild and can be used instead of the others.

\subsection*{Crafting Cards}
    To craft a card, Activate \textbf{Crafting Pieces} (faction specific) in the Clearings shown on card's bottom-left. If present, resolve \textbf{Immediate Effect}. If supply doesn't have indicated item, you can't craft card. If has \textbf{Persistent Effect}, may only have one of a given name.

\subsection*{Battling}
Pick clearing where you have any warriors. You are the attacker, choose another factions with any pieces there as defender.
    \begin{enumerateCustom}
        \item Roll dice. Attacker deals higher roll hits, defender deals lower roll. Maximum hits equal to number of warriors in battle.
        \item Remove both sides pieces at same time. Player taking hits picks what to remove. Must remove warriors before tokens or buildings.
    \end{enumerateCustom}

\subsubsection*{Extra Hits}
Some effects deal \textbf{Extra Hits}. Not limited by number of warriors in battle. If defender has no warriors in battle, attacker gets Extra Hit.

\subsubsection*{Ambush Cards}
Before rolling in battle, defender may play Ambush Card with Suit matching clearing to deal two hits immediately. Attacker may foil Ambush Card by also playing a matching Ambush Card. May be played even if Defenseless. Battle ends immediately if no attacker warriors present.

\subsection*{Removing Buildings / Tokens}
Whenever you remove an enemy building / token score a victory point.

\subsection*{Dominance Cards}
A Dominance card changes your victory condition when activated. If discarded or spent, don't discard Dominance cards. Instead place near map. Any player, during Daylight, may pick up a Dominance Card by spending a matching suit card.

\subsection*{Change Victory Condition}
If you have at least 10 victory points, may play Dominance Card into play area to Activate it. Remove score marker and place on card. For rest of game, may only win by meeting condition listed. Can't remove or replace Dominance Card.

\subsection*{Forming Coalition as Vagabond}
Vagabond may play Domination Card to form \textbf{Coalition} with other player with lowest score. Take Vagabond score marker and place on other factions board. Vagabond shares victory if other faction wins. If other Faction is Hostile, put marker on Indifference space.

\section*{Gameplay}
Players take turns in clockwise order. Each turn has 3 phases: Birdsong, Daylight, and Evening.

\end{multicols*}
\end{document}
