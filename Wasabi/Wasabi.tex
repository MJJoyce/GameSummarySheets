\documentclass[12pt]{article}

% Set landscape mode and custom margins for page while including space for the footer
\usepackage[includefoot, margin={0.5cm, 0.5cm}, landscape]{geometry}

% Used to reduce the spacing around section headers and titles.
% Format title headers by adjusting font size
% Set the spacing on all sides of the title to 0
\usepackage[compact]{titlesec}
\titleformat{\section}{\normalfont\bfseries}{\thesection}{1em}{}
\titleformat{\subsection}{\normalfont\bfseries}{\thesection}{0.2em}{}
\titlespacing*{\section}{0pt}{*0}{0pt}
\titlespacing*{\subsection}{0pt}{*0}{0pt}

% Reduce spacing around list items
\usepackage{enumitem}
\setlist{nolistsep}

% Set line spacing
\usepackage{setspace}
\singlespacing

% Create a multicolumn layout
% Set the amount of separation between columns
% Draw a vertical rule between the columns
\usepackage{multicol}
\setlength{\columnsep}{1cm}
\setlength{\columnseprule}{0.1pt}

% Add a frame around the content
\usepackage{mdframed}

% Custom footer (and header if I wanted)
\usepackage{fancyhdr}
\pagestyle{fancy}

% Set custom headers and footers for fancyhdr
\fancyhead{}
\fancyfoot[C]{Wasabi Summary Sheet v1.0}
\renewcommand{\headrulewidth}{0pt} 	% Remove horizontal rule from header
\renewcommand{\footrulewidth}{0pt}  % Remove horizontal rule from footer

% Custom frame style for mdframed
% The negative margin is needed to fix a weird spacing that I couldn't figure out
\mdfdefinestyle{customFrame}{%
    outerlinewidth = 0.4pt,
    innertopmargin = -0.3cm}

% Reduce whitespace in the enumerate list environment
\newenvironment{enumerateCustom}
{\begin{enumerate}
  \setlength{\itemsep}{1pt}
  \setlength{\parskip}{0pt}
  \setlength{\parsep}{0pt}}
{\end{enumerate}}

% Reduce whitespace in the itemize list environment
\newenvironment{itemizeCustom}
{\begin{itemize}
  \setlength{\itemsep}{1pt}
  \setlength{\parskip}{0pt}
  \setlength{\parsep}{0pt}}
{\end{itemize}}

\begin{document}
\begin{mdframed}[style = customFrame]
\begin{multicols*}{2}

\section*{Setup}
\begin{itemizeCustom}
    \item Sort ingredients by type and stack near board. Sort by background color categories.
    \item Separate recipe strips by length, shuffle, place face down near board.
    \item Place action cards face up near board:
        \begin{itemizeCustom}
            \item 2-player: 1 of each type
            \item 3-player: 2 Wasabi, 2 Spicy, 1 Chop, 1 Switch, and 1 Stack
            \item 4-player: All
        \end{itemizeCustom}
    \item Place Wasabi cubes in supply near board
    \item Give each player a screen, a bowl, and set of challenge tokens. Place challenge tokens in front of player length side up.
    \item Starting with first player, grab 3 starting tiles (cannot be Rice, Maki, or a Unique Ingredient) and give them to the player to left. Continue with next player.
    \item Take 3 recipes of whatever length and place them in your screen. Continue with next player.
\end{itemizeCustom}

\section*{Goal of the Game}
Prepare the best sushi, finish your recipes, and meet your challenges -- all with as much style as possible.

\section*{Player Turn}
\subsection*{Play an Ingredient Tile and Optionally an Action Card}
You must place one Ingredient tile onto an unoccupied square on the board. If you placed a tile that completes one of your recipes and you have an unassigned Challenge Token of that length, remove recipe from screen, place face-up in front of you, and assign a challenge token of the correct length to it. If completed with style, take the specified number of Wasabi cubes.

Note, you do not need to complete recipes in the order specified, however, if you do you will complete it with style.

You may also play an Action Card. Depending on the card you might play it before or after you lay a tile.

\subsection*{Prepare for Next Turn}
\begin{enumerateCustom}
    \item Collect 1 Action Card for each recipe you completed. You cannot take the same type of card that you played this turn. You may never have more than 2 Action Cards. You may discard a card to the Kitchen to make room for a newly earned one.
    \item If you have fewer than 3 ingredient tiles, draw up to 3.
    \item If you have fewer than 3 recipes, draw up to 3 (size is your choice).
\end{enumerateCustom}

\section*{End Game}
\begin{itemizeCustom}
    \item The game ends if a player assigns all 10 of their challenge tokens. No points are counted in this situation as this player immediately wins. 
    \item If at the end of a player's turn, all the spaces on the board are filled. Count assigned challenge token points + 1 for each Wasabi cube.
\end{itemizeCustom}

\section*{Action Cards}
\subsection*{Stack!}
Play before playing ingredient tile. You may now place the ingredient on top of another ingredient on the board.

\subsection*{Spicy!}
Play either before or after playing ingredient tile. Play a total of 2 ingredient tiles this turn.

\subsection*{Switch!}
Play either before or after playing ingredient tile. Swap two adjacent squares on the board. This can complete a recipe.

\subsection*{Chop!}
Play either before or after playing ingredient tile. Take any 1 ingredient off the board (only take the top of a stack). You may return it to the pantry or play it as your ingredient for this turn (if you haven't already done so). You may complete a recipe by removing the top of a stack with Chop!

\subsection*{Wasabi!}
Play either before or after playing ingredient tile. Place the card onto the board covering 4 squares (may or may not be occupied). Take 1 Wasabi cube from supply.
\begin{itemizeCustom}
    \item No ingredient may be played into the covered squares.
    \item No covered ingredient may be used to complete a recipe.
    \item No covered square may be affected by any other action card.
    \item Covered square counts for end game condition.
    \item Only comes off the board when any player chooses to take a Wasabi card as a reward for completing a recipe.
\end{itemizeCustom}

\end{multicols*}
\end{mdframed}
\end{document}
