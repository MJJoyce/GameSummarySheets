\documentclass[12pt]{article}

% Set landscape mode and custom margins for page while including space for the footer
\usepackage[includefoot, margin={0.5cm, 0.5cm}, landscape]{geometry}

% Used to reduce the spacing around section headers and titles.
% Format title headers by adjusting font size
% Set the spacing on all sides of the title to 0
\usepackage[compact]{titlesec}
\titleformat{\section}{\normalfont\bfseries}{\thesection}{1em}{}
\titlespacing*{\section}{0pt}{*0}{0pt}

\titleformat{\subsection}{\normalfont\bfseries}{\thesection}{0.3em}{}
\titlespacing*{\subsection}{0pt}{*0}{0pt}

% Reduce spacing around list items
\usepackage{enumitem}
\setlist{nolistsep}

% Set line spacing
\usepackage{setspace}
\singlespacing

% Create a multicolumn layout
% Set the amount of separation between columns
% Draw a vertical rule between the columns
\usepackage{multicol}
\setlength{\columnsep}{1cm}
\setlength{\columnseprule}{0.1pt}

% Add a frame around the content
\usepackage{mdframed}

% Custom footer (and header if I wanted)
\usepackage{fancyhdr}
\pagestyle{fancy}

% Set custom headers and footers for fancyhdr
\fancyhead{}
\fancyfoot[C]{Lord of the Rings: The Card Game Summary Sheet v1.0}
\renewcommand{\headrulewidth}{0pt} 	% Remove horizontal rule from header
\renewcommand{\footrulewidth}{0pt}  % Remove horizontal rule from footer

% Custom frame style for mdframed
% The negative margin is needed to fix a weird spacing that I couldn't figure out
\mdfdefinestyle{customFrame}{%
    outerlinewidth = 0.4pt,
    innertopmargin = -0.3cm}

% Reduce whitespace in the enumerate list environment
\newenvironment{enumerateCustom}
{\begin{enumerate}
  \setlength{\itemsep}{1pt}
  \setlength{\parskip}{0pt}
  \setlength{\parsep}{0pt}}
{\end{enumerate}}

% Reduce whitespace in the itemize list environment
\newenvironment{itemizeCustom}
{\begin{itemize}
  \setlength{\itemsep}{1pt}
  \setlength{\parskip}{0pt}
  \setlength{\parsep}{0pt}}
{\end{itemize}}

\begin{document}
\begin{multicols*}{2}

\section*{Setup}
\begin{enumerateCustom}
	\item Each player should pick and shuffle their deck.
	\item Each player places his heroes in front of him and sets his threat tracker to their total cost.
	\item Set out all the tokens.
	\item Draw starting hand of 6 cards. Players may mulligan once.
	\item Place quest cards in sequential order, A-side up, to make the Quest deck.
	\item Follow any Scenario setup instructions.
\end{enumerateCustom}

\section*{Round Sequence}
The game is played out over a series of Rounds, each of which has 7 Phases.

\subsection*{Phase 1: Resource}
\begin{itemizeCustom}
	\item Each player adds 1 resource token to each heroes' Resource Pool. A hero's resources can only be used to pay for cards from its Sphere of Influence.
	\item Each player draws 1 card. If draw pile is empty, \emph{do not} draw.
\end{itemizeCustom}

\subsection*{Phase 2: Planning}
In player order, play an Ally or Attachment cards. This is the \emph{only} phase when they can be played.

\subsection*{Phase 3: Quest}
Players now attempt to make progress on the current stage of their quest. Players may play event cards and take actions at the beginning and the end of each step
\begin{enumerateCustom}
	\item \textbf{Commit Characters:} Players commit heroes to the quest as a team in player order. Exhaust any heroes that are committed to the quest.
	\item \textbf{Staging:}  Reveal one card per player from the Encounter deck. Reveal them one at a time and resolve any effects before drawing additional cards. Enemy and location cards are placed in the Staginng area. Treachery cards are resolved and discarded (unless otherwise indicated).
	\item \textbf{Quest Resolution:} Compare combined Willpower of committed characters against combined Threat of all cards in Staging area.
		\begin{itemizeCustom}
			\item Willpower $>$ Threat: Add the difference as progress tokens to the current quest (Active Location card if present before quest).
			\item Willpower $<$ Threat: Add the difference to each player's Threat Tracker.
			\item Willpower $=$ Threat: Do nothing.
		\end{itemizeCustom}
\end{enumerateCustom}

\subsection*{Phase 4: Travel}
Players may travel as a group to a Location card in the Staging area. Move the location card from the Staging area and place it next to the Quest deck. May only have one active Location and cannot travel if there is an active Location card. An active Location card no longer contributes its Threat to the Staging area. Any quest progress is placed onto the active Location card before the Quest card.

\subsection*{Phase 5: Encounter}
\begin{itemizeCustom}
	\item First, each player may choose to engage one enemy in the Staging area. Move the enemy to in front of the engaging player.
	\item In player order, compare the player's Threat level with each enemy in the Staging area. The enemy with the highest Engagement cost that is \emph{equal to or lower} than the player's Threat engages the player. Move it to in front of the player. Then the next player does the same. Do this until no enemy can engage a player.
\end{itemizeCustom}

\subsection*{Phase 6: Combat}
Deal 1 face down Shadow card to each engaged enemy from the Encounter deck. Deal cards in player order and from highest to lowest enemy engagement cost. If the Engagement deck runs out, \emph{do not shuffle the discard pile}. That is only done during the Questing phase!

\noindent
\textbf{Resolve enemy attacks.} The player can resolve enemy attacks in the order of their choosing. 
\begin{itemizeCustom}
	\item Pick an enemy attack to resolve.
	\item Next, declare a defender and exhaust it. Only one character can be declared a defender against each attacking enemy. A player may also choose to not defend an attack. 
	\item Now, flip over the Shadow card and resolve its Shadow effect (It's the text below the black bar). 
	\item Damage dealt to the defending character is the difference between the defending character's defense value and the attacking character's attack value.
	\item If an attack is undefended, all its damage must be assigned to a single hero. The defense value has no affect in this case.
\end{itemizeCustom}

\noindent
\textbf{Attacking enemies.} Each player may declare attacks against engaged enemies. 
\begin{itemizeCustom}
	\item Declare target of attack. Declare any number of attackers and exhaust them.
	\item Add up total attack values.
	\item Damage dealt is the difference between the total attack values and the defense of the target.
\end{itemizeCustom}

\section*{End Game}

\end{multicols*}
\end{document}
