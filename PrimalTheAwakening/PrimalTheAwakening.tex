\documentclass[12pt]{article}

% Set landscape mode and custom margins for page while including space for the footer
\usepackage[includefoot, margin={0.5cm, 0.5cm}, landscape]{geometry}

\usepackage{graphicx}

% Used to reduce the spacing around section headers and titles.
% Format title headers by adjusting font size
% Set the spacing on all sides of the title to 0
\usepackage[explicit]{titlesec}
\titleformat{\section}{\large\bfseries}{\thesection}{1em}{\underline{#1}}
\titleformat{\subsection}{\normalsize\bfseries}{\thesubsection}{1em}{#1}
\titleformat{\subsubsection}{\small\bfseries}{\thesubsubsection}{1em}{#1}
\titlespacing*{\section}{0pt}{*0}{0pt}
\titlespacing*{\subsection}{0pt}{*0}{0pt}
\titlespacing*{\subsubsection}{0pt}{*0}{0pt}

% Reduce spacing around list items
\usepackage{enumitem}
\setlist{nolistsep}

% Set line spacing
\usepackage{setspace}
\singlespacing

% Create a multicolumn layout
% Set the amount of separation between columns
% Draw a vertical rule between the columns
\usepackage{multicol}
\setlength{\columnsep}{1cm}
\setlength{\columnseprule}{0.1pt}

% Add a frame around the content
\usepackage{mdframed}

% Remove paragraph indent
\setlength{\parindent}{0pt}

% Custom footer (and header if I wanted)
\usepackage{fancyhdr}
\pagestyle{fancy}

% Set custom headers and footers for fancyhdr
\fancyhead{}
\fancyfoot[C]{Primal: The Awakening Summary Sheet v1.0}
\renewcommand{\headrulewidth}{0pt} 	% Remove horizontal rule from header
\renewcommand{\footrulewidth}{0pt}  % Remove horizontal rule from footer

% Custom frame style for mdframed
% The negative margin is needed to fix a weird spacing that I couldn't figure out
\mdfdefinestyle{customFrame}{%
    outerlinewidth = 0.4pt,
    innertopmargin = -0.3cm}

\mdfdefinestyle{SummaryCard}{
    linecolor=black,
    outerlinewidth=2pt,
    roundcorner=20pt,
    innertopmargin=4pt,
    innerbottommargin=4pt,
    innerrightmargin=4pt,
    innerleftmargin=4pt,
    leftmargin = 4pt,
    rightmargin = 4pt,
    frametitlealignment=\center
}

% Reduce whitespace in the enumerate list environment
\newenvironment{enumerateCustom}
{\begin{enumerate}
  \setlength{\itemsep}{1pt}
  \setlength{\parskip}{0pt}
  \setlength{\parsep}{0pt}}
{\end{enumerate}}

% Reduce whitespace in the itemize list environment
\newenvironment{itemizeCustom}
{\begin{itemize}
  \setlength{\itemsep}{1pt}
  \setlength{\parskip}{0pt}
  \setlength{\parsep}{0pt}}
{\end{itemize}}

\begin{document}

\begin{mdframed}[style=SummaryCard, align=center, userdefinedwidth=40em, frametitle={Round Order Guide}]

    \begin{enumerateCustom}
        \item Each player may resolve one \textbf{Consume} ability.
        \item Monster Upkeep
            \begin{itemizeCustom}
                \item Replace behavior card with lowest number (or all tied cards).
                \item Gain 1 per player (1\includegraphics[scale=0.40]{images/per_player.png}) \textbf{struggle} tokens.
                    \begin{itemizeCustom}
                        \item If $\ge$ 3\includegraphics[scale=0.40]{images/per_player.png}, \textbf{Unleash}
                    \end{itemizeCustom}
            \end{itemizeCustom}

        \item Player Turns: Clockwise turn order starting with \textbf{aggro} token player.
            \begin{itemizeCustom}
                \item Movement Phase
                    \begin{itemizeCustom}
                        \item May spend 1 \textbf{stamina} to move to adjacent sector.
                            \begin{itemizeCustom}
                                \item \textit{Remember, remove threatened status when you move.}
                            \end{itemizeCustom}
                        \item If don't move: take \textbf{threatened} token.
                    \end{itemizeCustom}
                \item Action Phase: Perform actions in any order and count.
                    \begin{itemizeCustom}
                        \item Play action card.
                            \begin{itemizeCustom}
                                \item Play up to 5 action cards into your \textbf {sequence}. Resolve card ability text and then specific card color effect.
                                    \begin{itemizeCustom}
                                        \item Take \textbf{aggro} token immediately when played, if relevant.
                                        \item Player chooses order to resolve simultaneous triggered abilities on played cards.
                                    \end{itemizeCustom}
                            \end{itemizeCustom}
                        \item Activate \textbf{Action} ability
                        \item Revive (limit once per turn)
                            \begin{itemizeCustom}
                                \item If in sector with KO'd player, spend 2 stamina. Flip red KO token to black side or discard black KO token and player rises.
                            \end{itemizeCustom}

                        \textbf{End of Phase:} Gain \textbf{stamina} token if $\ge 2$ cards in hand.
                    \end{itemizeCustom}
                \item Attrition Phase
                    \begin{itemizeCustom}
                        \item Reveal 1 attrition card (2, pick highest if \textbf{threatened})
                        \item Take \textbf{attrition damage} (\includegraphics[scale=0.30]{images/monster_damage.png}) if number of defense cards in sequence $<$ value. See damage value on bottom-right of stance card.
                    \end{itemizeCustom}
                \item End of Turn
                    \begin{itemizeCustom}
                        \item Discard sequence in order played
                        \item Draw / discard to hand limit (default 5)
                            \begin{itemizeCustom}
                                \item When deck is empty, suffer damage equal to weapon level and then reshuffle.
                            \end{itemizeCustom}
                        \item Rotate monster to face \textbf{aggro} player.
                    \end{itemizeCustom}
            \end{itemizeCustom}

        \item End of Round
            \begin{itemizeCustom}
                \item Resolve "at the end of the round" effects
                \item Advance round marker.
            \end{itemizeCustom}
    \end{enumerateCustom}
\end{mdframed}

\pagebreak

\begin{mdframed}[style=SummaryCard, align=center, userdefinedwidth=35em, frametitle={Important Keywords (Mirah)}]
    \begin{description}
        \item[Assist] During another player’s turn, either before or after any action, you may discard an Assist card. The active player draws 1. Limit once per turn.
        \item[Blind] When you inflict Blind, choose a peril card in play. That peril is considered to be blank until the start of your next turn.
        \item[Focused Mastery] \includegraphics[scale=0.30]{images/mastery_icon.png} If card ability text begins with this icon, it means you can only resolve the ability if your mastery card is focused.
        \item[Recycle] When you are instructed to Recycle X, you may discard X cards of your choice from your hand to draw X.
        \item[Stealth] When you play an action card with the keyword Stealth in your sequence, that card do not trigger reaction icons containing color or card type icons.
        \item[Volley] You may discard X cards from the top of your deck and deal damage to the monster equal to your weapon level for each offensive card (red / blue) discarded this way.
        \item[Weapon Icon] \includegraphics[scale=0.20]{images/weapon_level.png} These are the class icons that represent the different hunter classes. They refer to your weapon level (top right corner of weapon card).
    \end{description}
\end{mdframed}

\pagebreak

\begin{mdframed}[style=SummaryCard, align=center, userdefinedwidth=35em, frametitle={Important Keywords (Dareon)}]
    \begin{description}
        \item[Assist] During another player’s turn, either before or after any action, you may discard an Assist card. The active player draws 1. Limit once per turn.
        \item[Berserker] This is a keyword ability. When you play a card with the keyword Berserker in your sequence, you enter the berserker state and remain in that state until the end of your turn. While in the berserker state, draw 1 each time you play an attack card in your sequence.
        \item[Finisher] After you play a card with the keyword Finisher in your sequence, your Action phase ends
        \item[Focused Mastery] \includegraphics[scale=0.30]{images/mastery_icon.png} If card ability text begins with this icon, it means you can only resolve the ability if your mastery card is focused.
        \item[Resilience] After you suffer , you may reveal a card with the keyword Resilience from your hand to draw 1. Limit once per Resilience card per round.
        \item[Vulnerable] Inflict Vulnerable on the monster. While the monster is vulnerable, you may discard the vulnerable token to double a single source of damage. Vulnerability is removed at the end of the round.
        \item[Weapon Icon] \includegraphics[scale=0.20]{images/weapon_level.png} These are the class icons that represent the different hunter classes. They refer to your weapon level (top right corner of weapon card).
    \end{description}
\end{mdframed}

\pagebreak

\begin{mdframed}[style=SummaryCard, align=center, userdefinedwidth=35em, frametitle={Important Keywords (Thoreg)}]
    \begin{description}
        \item[Assist] During another player’s turn, either before or after any action, you may discard an Assist card. The active player draws 1. Limit once per turn.
        \item[Confuse] Give monster the confused token and turn monster to a sector of your choice. When the monster would activate a boost effect, discard the confused token instead and cancel that effect (still pay the boost cost). Remove at beginning of next round.
        \item[Focused Mastery] \includegraphics[scale=0.30]{images/mastery_icon.png} If card ability text begins with this icon, it means you can only resolve the ability if your mastery card is focused.
        \item[Recycle] When you are instructed to Recycle X, you may discard X cards of your choice from your hand to draw X.
        \item[Stealth] When you play an action card with the keyword Stealth in your sequence, that card do not trigger reaction icons containing color or card type icons.
        \item[Strain] Then the next time you refill your hand, draw one card fewer for each strain you have.
        \item[Stun] When you stun the monster, place stun token on an active behavior card. That card cannot be triggered this round. Then choose a player. They reveal up to two attack cards from their hand and deal weapon damage per revealed card. Remove after resolving "end of round" effects.
        \item[Weapon Icon] \includegraphics[scale=0.20]{images/weapon_level.png} These are the class icons that represent the different hunter classes. They refer to your weapon level (top right corner of weapon card).
    \end{description}
\end{mdframed}

\pagebreak

\begin{mdframed}[style=SummaryCard, align=center, userdefinedwidth=35em, frametitle={Important Keywords (Ljonar)}]
    \begin{description}
        \item[Assist] During another player’s turn, either before or after any action, you may discard an Assist card. The active player draws 1. Limit once per turn.
        \item[Confuse] Give monster the confused token and turn monster to a sector of your choice. When the monster would activate a boost effect, discard the confused token instead and cancel that effect (still pay the boost cost). Remove at beginning of next round.
        \item[Focused Mastery] \includegraphics[scale=0.30]{images/mastery_icon.png} If card ability text begins with this icon, it means you can only resolve the ability if your mastery card is focused.
        \item[Resilience] After you suffer \includegraphics[scale=0.30]{images/monster_damage.png}, you may reveal a card with the keyword Resilience from your hand to draw 1. Limit once per Resilience card per round.
        \item[Taunt] During another player’s turn, discard a Taunt card. Active player draws 1, you take the aggro token, and turn monster to your sector. Limit once per turn.
        \item[Vulnerable] Inflict Vulnerable on the monster. While the monster is vulnerable, you may discard the vulnerable token to double a single source of damage. Vulnerability is removed at the end of the round.
        \item[Weapon Icon] \includegraphics[scale=0.20]{images/weapon_level.png} These are the class icons that represent the different hunter classes. They refer to your weapon level (top right corner of weapon card).
    \end{description}
\end{mdframed}

\pagebreak

%\begin{mdframed}[style=SummaryCard, align=center, userdefinedwidth=60em, frametitle={Important Triggers and Resolutions}]
\begin{multicols*}{2}

%\section*{Important Triggers and Resolutions}
\subsection*{Behavior Card Resolution}
When the effects of any event or card trigger a behavior card, completely resolve that effect before resolving the behavior card. When multiple behaviors are triggered (excluding rampage), reveal and resolve those cards fully, one at a time.  
    \begin{enumerateCustom}
        \item Apply behavior card effect and boost effect (if any).
        \item Resolve any rampage cards in play.
        \item Discard behavior card(s) and refill behavior slot.
    \end{enumerateCustom}

\subsection*{Card Resolution Timing}
Resolve simultaneous triggers with the following priority. If multiple cards have the same priority, the players decide the resolution order.
    \begin{enumerateCustom}
        \item Stance Cards
        \item Peril Cards
        \item Behavior Cards
        \item Other Cards
    \end{enumerateCustom}

\subsection*{Empty Player Deck}
Suffer damage equal to your weapon level (a.k.a., \textbf{fatigue damage}). Immediately reshuffle your deck.

\subsection*{Empty Behavior Deck}
Immediately reshuffle discard pile. Monster gains 1 struggle (a.k.a., \textbf{escalation}).

\subsection*{Knocked Out}
When a players total sustained damage $\ge$ total health (sum of equipped armor and helm HP) they are KO'd:
    \begin{enumerateCustom}
        \item If you're the active player, immediately end your turn (skipping all other steps).
        \item Remove all damage tokens from hunter board.
        \item Pass aggro token to first player (or next player in player order if they're KO'd).
        \item Discard all cards in hand and sequence.
        \item Place \textbf{deplete token} on either armor or helm. Must not be currently depleted. Depleted equipment does not contribute towards your HP.
        \item Add a \textbf{Would card} to discard pile. Shuffle deck.
        \item Lay hunter miniature on side to indicate KO.
        \item Take KO token, place on hunter board red side up.
        \item Turn monster to aggro player.
    \end{enumerateCustom}

    When it is your turn while knocked out, flip KO token to black side. If already on black side, the hunter \textbf{rises}:
    \begin{enumerateCustom}
        \item Discard KO token.
        \item Draw to hand size.
        \item Stand up miniature on board.
        \item Your turn ends.
    \end{enumerateCustom}

\subsection*{Unleash}
If struggle $\ge$ 3\includegraphics[scale=0.40]{images/per_player.png}:
    \begin{enumerateCustom}
        \item All players suffer \includegraphics[scale=0.30]{images/monster_damage.png}
        \item Remove all but 1\includegraphics[scale=0.40]{images/per_player.png} struggle tokens.
    \end{enumerateCustom}

\subsection*{Winning and Losing}
Scenarios end in the following ways:
    \begin{itemizeCustom}
        \item Players win immediately if they reduce the monsters health to 0
        \item Players lose if the end of round 10 is reached or when all players are simultaneously KO'd.
    \end{itemizeCustom}
%\section*{Goal of the Game}
%\section*{Player Turn}
%\section*{End Game}

\end{multicols*}
%\end{mdframed}
\end{document}
